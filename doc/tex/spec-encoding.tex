\subsection{\XCRYPTO encoding}
\label{sec:spec:encoding}

% =============================================================================

\subsubsection{Sub-word size}
\label{sec:spec:encoding:subword}

The instructions in some feature classes
(see, e.g., \REFSEC{sec:spec:instruction:2:4})
(semi-)independently operate on $w$-bit sub-words within 
source      $\GPR$ registers
and/or
destination $\GPR$ registers.
In some  cases 
$w$ is fixed by the instruction type
(e.g., \VERB[RV]{xc.ldr.b} per \REFSEC{sec:spec:instruction:xc.ldr.b}, where $w = 8$);
in other cases
$w$ is parameterised
(e.g., \VERB[RV]{xc.padd}  per \REFSEC{sec:spec:instruction:xc.padd}).
The encoding of instructions in the later case include \VERB[RV]{pw},
a field which maps to a sub-word size
$
w \in \SET{ 2, 4, 8, 16 }
$ 
per
\[
w ~~=~~ 2^{\VERB[RV]{pw}+1} ~~=~~ 1 \LSH ( \VERB[RV]{pw} + 1 )
\]
i.e., 
\[
\begin{array}{l c l@{\;}c@{\;}r c l@{\;}c@{\;}r}
\VERB[RV]{c} &\mapsto& \VERB[RV]{pw} &=& \RADIX{00}{2} &\mapsto& w &=&  2 \\
\VERB[RV]{n} &\mapsto& \VERB[RV]{pw} &=& \RADIX{01}{2} &\mapsto& w &=&  4 \\
\VERB[RV]{b} &\mapsto& \VERB[RV]{pw} &=& \RADIX{10}{2} &\mapsto& w &=&  8 \\
\VERB[RV]{h} &\mapsto& \VERB[RV]{pw} &=& \RADIX{11}{2} &\mapsto& w &=& 16 \\
\end{array}
\]
where a $1$-character mnemonic forms part of the instruction syntax.
Note that a selection of $w$ via \VERB[RV]{pw} implies that the number 
of sub-words will be
$
\RVXLEN / w .
$

% =============================================================================

\subsubsection{Register pairs}
\label{sec:spec:encoding:pair}

The instructions in some feature classes
(see, e.g., \REFSEC{sec:spec:instruction:2:5})
use 
more than $2$ source      $\GPR$ registers
and/or
more than $1$ destination $\GPR$ register.
Although doing so violates the consistency criteria outlined in 
\REFSEC{sec:bg:concept},
it aligns with selective demand for 
Multi-word Operand, Multi-word Result (MOMR) 
functionality as argued by
Lee et al.~\cite{SCARV:LeeYanShi:04}.
The encoding of such an instruction employs a limited instance of the
Register File Extension for Multi-word and Long-word Operation (RFEMLO)
proposal of 
Lee and Choi~\cite{SCARV:LeeCho:08}.
In short, a single compressed register address $x$ maps to
\[
\TUPLE{ ~ x \CONS 1, ~~ x \CONS 0 ~ } ,
\]
which we term a 
decompressed register address pair.  Consider an example, wlog. focused
on \VERB[RV]{rdm}, a compressed (destination) register address: the set 
of valid encodings is st.
\[
\begin{array}{l@{\;}c@{\;}r c c@{\;}l@{\;}c@{\;}r@{\;}c@{\;}l@{\;}c@{\;}r@{\;}c}
\VERB[RV]{rdm} &=     & \RADIX{0000}{2} &\mapsto& ( & \VERB[RV]{rd2} &=     &  1 &,& \VERB[RV]{rd1} &=     &  0 & ) \\
\VERB[RV]{rdm} &=     & \RADIX{0001}{2} &\mapsto& ( & \VERB[RV]{rd2} &=     &  3 &,& \VERB[RV]{rd1} &=     &  2 & ) \\
\VERB[RV]{rdm} &=     & \RADIX{0010}{2} &\mapsto& ( & \VERB[RV]{rd2} &=     &  5 &,& \VERB[RV]{rd1} &=     &  4 & ) \\
\VERB[RV]{rdm} &=     & \RADIX{0011}{2} &\mapsto& ( & \VERB[RV]{rd2} &=     &  7 &,& \VERB[RV]{rd1} &=     &  6 & ) \\
\VERB[RV]{rdm} &=     & \RADIX{0100}{2} &\mapsto& ( & \VERB[RV]{rd2} &=     &  9 &,& \VERB[RV]{rd1} &=     &  8 & ) \\
\VERB[RV]{rdm} &=     & \RADIX{0101}{2} &\mapsto& ( & \VERB[RV]{rd2} &=     & 11 &,& \VERB[RV]{rd1} &=     & 10 & ) \\
\VERB[RV]{rdm} &=     & \RADIX{0110}{2} &\mapsto& ( & \VERB[RV]{rd2} &=     & 13 &,& \VERB[RV]{rd1} &=     & 12 & ) \\
\VERB[RV]{rdm} &=     & \RADIX{0111}{2} &\mapsto& ( & \VERB[RV]{rd2} &=     & 15 &,& \VERB[RV]{rd1} &=     & 14 & ) \\
               &\vdots&                 &\vdots &   &                &\vdots&    & &                &\vdots&    &   \\
\VERB[RV]{rdm} &=     & \RADIX{1111}{2} &\mapsto& ( & \VERB[RV]{rd2} &=     & 31 &,& \VERB[RV]{rd1} &=     & 30 & ) \\
\end{array}
\]
The syntax of such instructions demands specification of the 
decompressed register address pair,
e.g., 
\[
\VERB[RV]{( x1, x0 )} \mapsto \TUPLE{ 1, 0 } ,
\]
which {\em must} adhere to the following rules: given a pair
$
\TUPLE{ x, y } ,
$
it {\em must} be true that
a) $x = y + 1$
   (the contiguous'ness rule),
   and
b) $x = 1 \pmod{2}$
   and 
   $y = 0 \pmod{2}$
   (the   odd-even'ness rule).
For example,
$\TUPLE{ 1, 0 }$ 
and 
$\TUPLE{ 3, 2 }$ are   valid,
$\TUPLE{ 3, 0 }$ is  invalid because it violates the contiguous'ness rule,
$\TUPLE{ 3, 1 }$ is  invalid because it violates the   odd-even'ness rule,
and
$\TUPLE{ 2, 3 }$ is  invalid because it violates the contiguous'ness rule 
                                       {\em and} the   odd-even'ness rule.

% =============================================================================
