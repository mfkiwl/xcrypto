% =============================================================================

\subsection{Guarantees}
\label{sec:bg:guarantee}

\begin{itemize}

\item The  
       specification of \XCRYPTO~{\em must}
      guarantee to exclude
                   undefined 
      and
      implementation defined
      functionality or semantics,
      bar any controlled (i.e., well-motivated) instances.

      A raft of challenges stemming from partially defined behaviour in the
      C programming language, and associated compilers, is well documented
      (see, e.g.,~\cite[Section 2.1]{SCARV:SimChiAnd:18}).
      These are particularly problematic for high-assurance software, where
      precise control over all aspects of execution behaviour forms a basis
      for guarantees wrt. security
      (see, e.g., various counterexamples such as~\cite{SCARV:KPVV:16}).
      A similar argument can obviously be made in relation to the platform 
      on which resulting software is executed, and, as such, this guarantee
      is intended to yield the greatest possible control over and therefore
      transparency wrt. execution behaviour.

\item An  
      implementation of \XCRYPTO~{\em must}
      a) guarantee that all computational 
         (i.e., {\em excluding} memory access) 
         instructions will exhibit constant (or data-oblivious) execution 
         latency,
         and
      b) clearly quote said constant in associated documentation.

      Note that exclusion of memory access stems naturally from the remit
      of \XCRYPTO as an ISE: it is impossible to control components beyond 
      the memory interface 
      (e.g., lower-level caches) 
      in a general way, which then limits the extent to which any general
      guarantee wrt. execution latency can be made.  

\end{itemize}

% =============================================================================
