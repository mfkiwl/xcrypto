\subsection{\XCRYPTO state}
\label{sec:spec:state}

% =============================================================================

\subsubsection{Class-$1$:   baseline}
\label{sec:spec:state:1}

\XCRYPTO adds a single user level CSR, the $\SPR{uxcrypto}$ register.
Figure \ref{fig:uxcrypto:bits} shows the bitfield layout of the register,
while table \ref{tab:uxcrypto:fields} describes the purpose of each field.

$\SPR{uxcrypto}$ provisionally exists at CSR address $0x800$.

\begin{figure}[h]
\begin{center}
\begin{bytefield}[bitwidth={1.4em},bitheight={8.0ex},endianness=big]{32}
\bitheader{0-31}               
\\
  \bitbox{ 8}{\rotatebox{90}{\small$\ID{b[1]}     $}}
& \bitbox{ 8}{\rotatebox{90}{\small$\ID{b[0]}     $}}
& \bitbox{ 4}{\rule{\width}{\height}}
& \bitbox{ 1}{\rotatebox{90}{\small$\ID{BIT }     $}}
& \bitbox{ 1}{\rotatebox{90}{\small$\ID{MEM }     $}}
& \bitbox{ 1}{\rotatebox{90}{\small$\ID{RNG }     $}}
& \bitbox{ 1}{\rotatebox{90}{\small$\ID{PACK}_{16}$}}
& \bitbox{ 1}{\rotatebox{90}{\small$\ID{PACK}_{8} $}}
& \bitbox{ 1}{\rotatebox{90}{\small$\ID{PACK}_{4} $}}
& \bitbox{ 1}{\rotatebox{90}{\small$\ID{PACK}_{2} $}}
& \bitbox{ 1}{\rotatebox{90}{\small$\ID{MP  }     $}}
& \bitbox{ 1}{\rotatebox{90}{\small$\ID{SHA3}     $}}
& \bitbox{ 1}{\rotatebox{90}{\small$\ID{SHA2}     $}}
& \bitbox{ 1}{\rotatebox{90}{\small$\ID{AES }     $}}
& \bitbox{ 1}{\rotatebox{90}{\small$\ID{CT  }     $}}
\end{bytefield}
\end{center}
\caption{A diagrammatic description of the $\SPR{uxcrypto}$ register;
blanked regions are reserved, so yield zero when the register is read.
Note that we force implementations to yield zero, rather than requiring
software to simply ignore these bits (as happens in the base RISC-V
specification) to remove any possibility of implementation specific
behaviour.}
\label{fig:uxcrypto:bits}
\end{figure}


\begin{table}[h]
\begin{center}
\begin{tabular}{|l|c|c|l|}
\hline
Field        & Index  & Access & Description                                                                         \\ 
\hline
$\ID{b[1]} $ & $ 31:24$ & R/W & Lookup table field used by \VERB[RV]{xc.bop} (\REFSEC{sec:spec:instruction:xc.bop}) \\
$\ID{b[0]} $ & $ 23:16$ & R/W & Lookup table field used by \VERB[RV]{xc.bop} (\REFSEC{sec:spec:instruction:xc.bop}) \\
$\ID{BIT }     $ & $11$ & R & Is the Bitwise class of instructions supported? \\
$\ID{MEM }     $ & $10$ & R & Is the Memory class of instructions supported? \\
$\ID{RNG }     $ & $ 9$ & R & Is the Randomness class of instructions supported? \\
$\ID{PACK}_{16}$ & $ 8$ & R & Is the pack width 16 class of instructions supported? \\
$\ID{PACK}_{8} $ & $ 7$ & R & Is the pack width 8  class of instructions supported? \\
$\ID{PACK}_{4} $ & $ 6$ & R & Is the pack width 4  class of instructions supported? \\
$\ID{PACK}_{2} $ & $ 5$ & R & Is the pack width 2  class of instructions supported? \\
$\ID{MP  }     $ & $ 4$ & R & Is the Multi-precision class of instructions supported? \\
$\ID{SHA3}     $ & $ 3$ & R & Is the SHA3 class of instructions supported? \\
$\ID{SHA2}     $ & $ 2$ & R & Is the SHA2 class of instructions supported? \\
$\ID{AES }     $ & $ 1$ & R & Is the AES class of instructions supported? \\
$\ID{CT} $ & $     0$ & R/W & Constant time indicator bit. \\
\hline
\end{tabular}
\end{center}
\caption{A tabular description of the $\SPR{uxcrypto}$ register.}
\label{tab:uxcrypto:fields}
\end{table}

The $\SPR{uxcrypto.CT}$ bit is used to control the timing and execution
behaviour of certain instructions.
It should be used to indicate that the currently running code must run
in constant time, or more accurately, time independent of the data
inputs.
Integer arithmetic instructions such as multiply, which might have an
early-out mechanism are the principle target of the $\SPR{uxcrypto.CT}$ bit.
\begin{itemize}
\item Memory accesses are not required to enforce constant time operation,
      though this is a potential side channel which must be mitigated.
\item Any integer arithmetic instructions with an early-out mechanism must
      run in constant time while the $\SPR{uxcrypto.CT}$ bit is set.
      They may or may not run in constant time when the bit is clear.
\end{itemize}

% =============================================================================

\subsubsection{Class-$2.6$: leakage}
\label{sec:spec:state:2:6}

% TODO: add FENL content

% =============================================================================
